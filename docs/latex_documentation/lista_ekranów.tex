\documentclass[a4paper, 11pt]{article}

% --- Pakiety do obsługi języka polskiego ---
\usepackage[utf8]{inputenc}
\usepackage[T1]{fontenc}
\usepackage[polish]{babel}
\usepackage{graphicx}

% --- Ustawienia strony ---
\usepackage[a4paper, margin=1in]{geometry}

% --- Lepsza kontrola nad listami (opcjonalnie, ale polecane) ---
\usepackage{enumitem}

% --- Tytuł dokumentu ---
\title{Specyfikacja ekranów i funkcjonalności LekturAI}
\date{} % Usuwa datę

\begin{document}

\maketitle
\tableofcontents % Automatyczny spis treści
\newpage

\section{Login Screen}
\begin{itemize}
    \item Logo "LekturAI"
    \item \textbf{Textfield:} email
    \item \textbf{Textfield:} hasło
    \item \textbf{Button:} zaloguj się
    \item \textbf{Button:} zarejestruj się
\end{itemize}
\includegraphics[width=0.5\textwidth, keepaspectratio]{images/login_screen.jpg}


\section{View / Edit Profile Screen}
\begin{itemize}
    \item Liczba punktów
    \item Zmiana hasła:
        \begin{itemize}
            \item stare hasło
            \item nowe hasło
            \item powtórz nowe hasło
        \end{itemize}
    \item Wybór swojego Miasta
    \item Wybór swojej szkoły z listy bazy danych.
    \item Wpisanie numeru klasy (np. 3, 4b).
    \item Ustawianie częstotliwości powiadomień (przypomnień o nauce).
\end{itemize}

\subsection{Prerequisities}
\begin{itemize}
    \item Baza danych szkół - jest plik otwarty, można pobrać i wrzucić do bazy danych.
    \item Szukanie szkół po mieście i nazwie.
\end{itemize}

\section{Test z lektur screen}
\begin{itemize}
    \item Wybór lektury z listy (po nazwie).
    \item Wybór zakresu rozdziałów.
    \item Test w formie pytań zamkniętych i otwartych.
    \item Przycisk "zakończ" $\rightarrow$ przekierowuje do ekranu wyników, gdzie jest wynik procentowy i wszystkie odpowiedzi poprawione, z komentarzami, dlaczego tyle punktów.
\end{itemize}

\section{Rozwiąż zadanie maturalne Screen}
\begin{itemize}
    \item Wybór epoki.
    \item Losowane pytanie ze zbioru pytań.
    \item Przycisk "zakończ" $\rightarrow$ ekran końcowy.
\end{itemize}

\section{Asystent rozprawki Screen}
\begin{itemize}
    \item \textbf{Textfield:} Polecenie (Temat).
    \item \textbf{Checkboxy:} wybór kontekstów, liczby argumentów itp.
    \item Przycisk "znajdź" $\rightarrow$ pokazuje listę kontekstów, argumentów itp.
\end{itemize}

\section{Historia testów, pytań Screen}
\begin{itemize}
    \item Lista rozwiązanych testów i pytań.
    \item Można wejść w test i zobaczyć listę pytań, swoich odpowiedzi oraz błędów.
\end{itemize}

\section{Home Screen}
\begin{itemize}
    \item Menu nawigacyjne:
        \begin{itemize}
            \item Test z lektur
            \item Zadania maturalne
            \item Asystent Rozprawki
            \item Historia pytań
            \item Konto
        \end{itemize}
    \item Strike nauki: 23 dni
    \item Wykres z punktami lub czasem (np. ile minut poświęcono na naukę każdego dnia w ostatnim tygodniu).
    \item \textbf{Floating Button:} ChatBot
\end{itemize}

\includegraphics[width=0.5\textwidth, keepaspectratio]{images/home_screen.jpg}

\end{document}