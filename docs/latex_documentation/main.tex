\documentclass{article}
\usepackage{graphicx} % Required for inserting images
\usepackage[polish]{babel}
\usepackage[utf8]{inputenc}
\usepackage[T1]{fontenc}
\title{Features PZ}
\date{\today}

\begin{document}

\maketitle


\section*{Must Have}

\subsection*{Ocenianie zadań / wypowiedzi pisemnych zgodnie z kluczem odpowiedzi}
\begin{itemize}
    \item Przedstawienie klucza odpowiedzi do danego zadania
    \item Porównanie odpowiedzi z odpowiednimi fragmentami klucza
    \item Wystawienie oceny względem klucza
\end{itemize}

\subsection*{Tworzenie pytań z wiedzy lektur}
\begin{itemize}
    \item Tworzenie zestawu pytań do danej lektury
    \item Tworzenie pytań do max podanego rozdziału w lekturze
\end{itemize}

\subsection*{Historia pytań i ocen}
\begin{itemize}
    \item Wyświetlanie poprzednich pytań i odpowiedzi na nie
    \item Statystyki poprawności odpowiedzi, ilości błędów, itd.
\end{itemize}

\subsection*{Obsługa kont użytkowników}
\begin{itemize}
    \item Rejestracja konta
    \item Logowanie użytkownika
    \item Przechowywanie danych użytkownika
    \item Rola użytkownika / grupa użytkownika (klasa, szkoła)
\end{itemize}

\section*{Should Have}

\subsection*{Elementy grywalizacji}
\begin{itemize}
    \item System odznak (np. „Mistrz Mickiewicza")
    \item „Seria dni nauki"
    \item Tryb rywalizacji 1v1 --- quiz na czas przeciwko innym uczniom
    \item Misje tygodniowe („napisz 2 rozprawki w tym tygodniu", „rozwiąż test z \textit{Lalki}")
\end{itemize}

\subsection*{Codzienne przypomnienia o nauce}
\begin{itemize}
    \item Powiadomienia push
    \item Możliwość ustawienia częstotliwości powiadomień
\end{itemize}

\subsection*{Grupy klasowe}
\begin{itemize}
    \item Znajdowanie swojej klasy, porównywanie wyników, streaków...
\end{itemize}

\subsection*{Wyszukiwanie kontekstów do tezy / tematu}
\begin{itemize}
    \item Szukanie według epok
    \item Lektury obowiązkowe / nieobowiązkowe
    \item (Could have) Wyzwanie szukania kontekstu (Bot zadaje kontekst, użytkownik próbuje wskazać lekturę, bot ocenia poprawność kontekstu)
\end{itemize}

\subsection*{Chatbot odpowiadający na pytania}
\begin{itemize}
    \item Powinien w miarę możliwości odsyłać do źródeł
    \item Linki do rozdziału źródłowego
\end{itemize}

\section*{Could Have}
\subsection*{Spersonalizowany plan nauki}

\subsection*{Profil bohatera}
\begin{itemize}
    \item Przybliżanie cech fizycznych i cech charakteru bohatera
\end{itemize}

\subsection*{Opracowanie pytań jawnych}
\begin{itemize}
    \item Odpowiedzi z dokładnością do faktów z lektur (bez interpretacji i kontekstów)
    \item Sugestie innych lektur z danej epoki/stylu literackiego jako kontekstów
\end{itemize}

\subsection*{Symulator matury ustnej}

\subsection*{Ocenianie zadań, matur, sprawdzianów ze zdjęcia}

\end{document}
